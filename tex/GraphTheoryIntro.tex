\setchapterpreamble[u]{\margintoc}
\chapter{Некоторые сведения из теории графов}
\label{chpt:GraphTheoryIntro}

В данном разделе мы дадим определения базовым понятиям из теории графов, рассмотрим несколько классических задач из области анализа графов и алгоритмы их решения.
Кроме этого, поговорим о связи между линейной алгеброй и некоторыми задачами анализа графов.
Всё это понадобится нам при последующей работе.

\section{Основные определения}

\begin{definition}[Помеченный ориентированный граф]
    \emph{Помеченный ориентированный граф} $\mbfscrG = \langle V, E, L \rangle$, где $V$~--- конечное множество вершин, $E$~--- конечное множество рёбер, т.ч. $E \subseteq V \times L \times V$, $L$~--- конечное множество меток на рёбрах.
    В некоторых случаях метки называют \emph{весами}%
    \sidenote{Весами метки называют, как правило, тогда, когда они берутся из какого-либо числового множества, например $\BbbR$ или $\BbbN$.}
    и тогда говорят о \emph{взвешенном} графе.
\end{definition}

\begin{definition}[Помеченный неориентированный граф]
    В случае, если для любого ребра $(u, l, v)$ в графе также содержится ребро $(v, l, u)$, говорят, что граф \emph{неориентированный}.
\end{definition}

\marginnote{
    \begin{example}
        Пусть дан граф
        \begin{align*}
            \mbfscrG_1 = \langle V & =\{0, 1, 2, 3\},                    \\
            E                      & =\{(0, a, 1), (1, a, 2), (2, a, 0), \\
                                   & \ \ \ (2, b, 3), (3, b, 2)\},       \\
            L                      & =\{a, b\} \rangle.
        \end{align*}
        Графическое представление графа $\mbfscrG_1$:
        \begin{center}
            \begin{tikzpicture}[]
  % Apex angle 60 degrees
  \node[isosceles triangle,
      isosceles triangle apex angle=60,
      draw=none,fill=none,
      minimum size=2cm] (T60) at (3,0){};


  \node[state] (q_0) at (T60.right corner) {$0$};
  \node[state] (q_1) at (T60.left corner)  {$1$};
  \node[state] (q_2) at (T60.apex)         {$2$};
  \node[state] (q_3) [right=1.5cm of q_2]  {$3$};
  \path[->]
    (q_0) edge[left]  node {a} (q_1)
    (q_1) edge[above] node {a} (q_2)
          edge[bend left, above] node {\textbf{S}} (q_3)
    (q_2) edge[below] node {a} (q_0)
          edge[bend left, above]  node {b} (q_3)
    (q_3) edge[bend left, below]  node {b} (q_2);
\end{tikzpicture}
        \end{center}
        $(0, a, 1)$ и $(3,b,2)$~--- это рёбра графа $\mbfscrG_1$.
        При этом $(3, b, 2)$ и $(2, b, 3)$~--- это разные рёбра.
    \end{example}
}
В дальнейшем речь будет идти о конечных ориентированных помеченных графах.
Мы будем использовать термин \emph{граф} подразумевая именно конечный ориентированный помеченный граф, если только не оговорено противное.

Также мы будем считать, что все вершины занумерованы подряд с нуля.
То есть можно считать, что $V$~--- это отрезок $[0 \rng |V| - 1]$, где $|V|$~--- мощность множества $V$.

\begin{definition}[Путь]
    \emph{Путём} $\pi$ в графе $\mbfscrG$ будем называть последовательность рёбер такую, что для любых двух последовательных рёбер $e_1 = (u_1, l_1, v_1)$ и $e_2 = (u_2, l_2, v_2)$ в этой последовательности, конечная вершина первого ребра является начальной вершиной второго, то есть $v_1 = u_2$.
    Будем обозначать путь из вершины $v_0$ в вершину $v_n$ как $v_0 \pi v_n$.
    Иными совами,
    \[v_0 \pi v_n = e_0,e_1, \dots, e_{n-1} = (v_0, l_0, v_1),(v_1,l_1,v_2),\dots,(v_{n-1},l_{n-1},v_n).\]
\end{definition}
Часто для представления пути мы буем использовать следующие нотации:
\begin{center}
    \begin{tikzpicture}[on grid, auto]
  \node[state] (v_0)   {$v_0$};
  \node[state] (v_n) [right=2 of v_0] {$v_n$};
  \path[->]
  (v_0) edge [out=45] node {$\pi$} (v_n);
\end{tikzpicture}

\end{center}
или
\[
    v_0 \xrightarrow[]{l_0} v_1 \xrightarrow[]{l_1} v_2 \xrightarrow[]{l_2} \ldots \xrightarrow[]{l_{n-2}} v_{n-1} \xrightarrow[]{l_{n-1}} v_n.
\]

\begin{example}
    $(0,a,1), (1,a,2) = 0 \pi_1 2$~--- путь из вершины 0 в вершину 2 в графе $\mbfscrG_1$.
    При этом $(0, a, 1), (1, a, 2), (2, b, 3), (3, b, 2) = 0 \pi_2 2$~--- это тоже путь из вершины 0 в вершину 2 в графе $\mbfscrG_1$, но он не равен $0 \pi_1 2$.
\end{example}

Кроме того, нам потребуется отношение, отражающее факт существования пути между двумя вершинами.

\begin{definition}[Достижимость в графе]
    \label{def:reach}
    \emph{Отношение достижимости} в графе: $(v_i,v_j) \in P \iff \exists v_i \pi v_j$.
    \marginnote{TODO: Не надо ли множество $P$ ввести аккуратнее?}
\end{definition}

Отметим, что в некоторых задачах удобно считать по умолчанию, что $(v_i,v_i) \in P$, однако наше определение такого не допускает.
Исправить ситуацию можно явно добавив петли $(v_i,l,v_i)$ для всех вершин.

Один из способов задать граф~--- это задать его \emph{матрицу смежности}.

\begin{definition}[Матрица смежности]
    \emph{Матрица смежности} графа $\mbfscrG = \langle V, E, L \rangle$~--- это квадратная матрица $M$ размера $n \times n$, где $|V| = n$, построенная над коммутативным моноидом $\BbbG = (S, \circ: S \times S \to S)$, который конструируется следующим образом.
    \begin{enumerate}
        \item $L \subseteq S$.
        \item $\circ$~--- коммутативная бинарная операция.
        \item Существует $\Bbbzero \in (S \setminus L)$~--- нейтральный элемент относительно $\circ$.
    \end{enumerate}
    При этом $M[i,j] = \bigcirc_{(i, l, j) \in E}l$, где $\bigcirc_\varnothing = \Bbbzero$.
\end{definition}

Заметим, что наше определение матрицы смежности отличается от классического, в котором матрица является булевой и отражает лишь факт наличия хотя бы одного ребра.
То есть $M[i,j] = 1 \iff \exists e = (i,\_,j) \in E$.

\begin{marginfigure}
    \caption{Неориентированный граф}
    \label{gr:graph1}
    \begin{center}
        \begin{tikzpicture}[]
  % Apex angle 60 degrees
  \node[isosceles triangle,
      isosceles triangle apex angle=60,
      draw=none,fill=none,
      minimum size=2cm] (T60) at (3,0){};


  \node[state] (q_0) at (T60.right corner) {$0$};
  \node[state] (q_1) at (T60.left corner)  {$1$};
  \node[state] (q_2) at (T60.apex)         {$2$};
  \node[state] (q_3) [right=1.5cm of q_2]  {$3$};
  \path[->]
    (q_0) edge[left]   node {$\{A\}$} (q_1)
    (q_1) edge[above]  node[yshift=5pt] {$\{A\}$} (q_2)
    (q_2) edge[below]  node[yshift=-5pt] {$\{A\}$} (q_0)
    (q_2) edge[bend left, above]  node {$\{B\}$} (q_3)
    (q_3) edge[bend left, below]  node {$\{B\}$} (q_2);
\end{tikzpicture}
    \end{center}
\end{marginfigure}
\begin{example}[Пример матрицы смежности неориентированного графа]
    \label{exmpl:undirectedGraphMatrix}
    Пусть дан неориентированный граф (см. рисунок \ref{gr:graph1}).
    $\BbbG = (S, \circ)$ в этом случае конструируется следующим образом.
    Во-первых, придётся предположить, что $L$~--- множество с одним элементом, скажем $s$, и считать, что все рёбра помечены им%
    \sidenote{А раз все рёбра имеют одинаковый заранее известный вес, то можно его и не писать для каждого ребра при задании графа.
        Поэтому привычное нам изображение получается достаточно логичным.}.
    Далее, $S = L \cup{\{n\}} = \{s,n\}$, где $n$~--- нейтральный элемент относительно $\circ$.
    \marginnote{TODO: Здесь $n$~--- нейтральный из-за клэша с $e$ из алгебры?}
    Тогда $\circ$ можно определить поточечно следующим образом.
    \begin{itemize}
        \item $s \circ s = s$
        \item $s \circ n = n \circ s = s$
        \item $n \circ n = n$
    \end{itemize}

    Таким образом, матрица смежности данного графа выглядит следующим образом:
    \[
        \begin{pmatrix}
            n & s & s & n \\
            s & n & s & n \\
            s & s & n & s \\
            n & n & s & n
        \end{pmatrix},
    \]
    что может показаться несколько непривычным.
    Однако заметим, что построенная нами структура $\BbbG = (\{s,n\}, \circ)$ изоморфна $\BbbG' = (\{1,0\}, \lor)$.
    При переходе к $\BbbG'$ мы получим привычную нам булеву матрицу смежности:
    \[
        \begin{pmatrix}
            0 & 1 & 1 & 0 \\
            1 & 0 & 1 & 0 \\
            1 & 1 & 0 & 1 \\
            0 & 0 & 1 & 0
        \end{pmatrix}
    \]
    Заметим, что матрица смежности неориентированного графа всегда симметрична относительно главной диагонали.
\end{example}

\begin{marginfigure}
    \caption{Ориентированный граф}
    \label{gr:graph2}
    \begin{center}
        \begin{tikzpicture}[]
  % Apex angle 60 degrees
  \node[isosceles triangle,
      isosceles triangle apex angle=60,
      draw=none,fill=none,
      minimum size=2cm] (T60) at (3,0){};


  \node[state] (q_0) at (T60.right corner) {$0$};
  \node[state] (q_1) at (T60.left corner)  {$1$};
  \node[state] (q_2) at (T60.apex)         {$2$};
  \node[state] (q_3) [right=1.5cm of q_2]  {$3$};
  \path[->]
    (q_0) edge[left]  node {a} (q_1)
          edge[bend right, below] node {S} (q_2)
    (q_1) edge[above] node {a} (q_2)
          edge[bend left, above, out=30, in=135] node {S} (q_3)
    (q_2) edge[below] node {a} (q_0)
          edge[bend left, above]  node {b, \textbf{S}} (q_3)
    (q_3) edge[bend left, below]  node {b} (q_2);
\end{tikzpicture}
    \end{center}
\end{marginfigure}
\begin{example}[Пример матрицы смежности ориентированного графа]
    \label{example:diGraph}
    Дан ориентированный граф (см. рисунок \ref{gr:graph2}).
    Построить его булеву матрицу смежности можно применив рассуждения из предыдущего примера (\ref{exmpl:undirectedGraphMatrix}) и выглядеть она будет следующим образом:
    \[
        \begin{pmatrix}
            0 & 1 & 0 & 0 \\
            0 & 0 & 1 & 0 \\
            1 & 0 & 0 & 1 \\
            0 & 0 & 1 & 0
        \end{pmatrix}.
    \]
\end{example}

\begin{marginfigure}
    \caption{Помеченный граф}
    \label{gr:graph3}
    \begin{center}
        \begin{tikzpicture}[]
  % Apex angle 60 degrees
  \node[isosceles triangle,
      isosceles triangle apex angle=60,
      draw=none,fill=none,
      minimum size=2cm] (T60) at (3,0){};


  \node[state] (q_0) at (T60.right corner) {$0$};
  \node[state] (q_1) at (T60.left corner)  {$1$};
  \node[state] (q_2) at (T60.apex)         {$2$};
  \node[state] (q_3) [right=1.5cm of q_2]  {$3$};
  \path[->]
    (q_0) edge[left]  node {$\{a\}$} (q_1)
    (q_1) edge[above]  node {$\{a\}$} (q_2)
    (q_2) edge[below]  node {$\{a\}$} (q_0)
    (q_2) edge[bend left = 15, above]  node {$\{a\}$} (q_3)
    (q_2) edge[bend left = 65, above]  node {$\{b\}$} (q_3)
    (q_3) edge[bend left, below]  node {$\{b\}$} (q_2);
\end{tikzpicture}
    \end{center}
\end{marginfigure}
\begin{example}[Пример матрицы смежности помеченного графа]
    Пусть дан помеченный граф (см. рисунок \ref{gr:graph3}).
    В данном случае $L = \{\{a\},\{b\}\}$, а $\BbbG = ( \{\{a\},\{b\},\{a,b\},\varnothing\} ,\cup)$, где $\varnothing$~--- нейтральный элемент.
    Тогда матрица матрица смежности исходного графа выглядит следующим образом:
    \[
        \begin{pmatrix}
            \varnothing & \{a\}       & \varnothing & \varnothing \\
            \varnothing & \varnothing & \{a\}       & \varnothing \\
            \{a\}       & \varnothing & \varnothing & \{a,b\}     \\
            \varnothing & \varnothing & \{b\}       & \varnothing
        \end{pmatrix}.
    \]
\end{example}

Необходимо заметить, что свойства структуры $\BbbG$, а значит и детали её построения, зависят от задачи, в рамках которой рассматривается граф.
В примерах выше мы строили $\BbbG$ из некоторых общих соображений, не специфицируя решаемую задачу, стараясь получить ожидаемый результат.
Далее мы рассмотрим пример, в котором видно, как решаемая задача влияет на построение $\BbbG$.

\begin{marginfigure}
    \caption{Взвешенный граф}
    \label{gr:graph4}
    \begin{center}
        \begin{tikzpicture}[]
  % Apex angle 60 degrees
  \node[isosceles triangle,
      isosceles triangle apex angle=60,
      draw=none,fill=none,
      minimum size=2cm] (T60) at (3,0){};


  \node[state] (q_0) at (T60.right corner) {$0$};
  \node[state] (q_1) at (T60.left corner)  {$1$};
  \node[state] (q_2) at (T60.apex)         {$2$};
  \node[state] (q_3) [right=1.5cm of q_2]  {$3$};
  \path[->]
    (q_0) edge[left]  node {a} (q_1)
          edge[bend right, below] node {S} (q_2)
          edge[bend right, below, out=330, in=225] node {\textbf{S}} (q_3)
    (q_1) edge[above] node {a, S} (q_2)
          edge[bend left, above, out=30, in=135] node {S} (q_3)
    (q_2) edge[below] node {a} (q_0)
          edge[bend left, above]  node {b, S} (q_3)
    (q_3) edge[bend left, below]  node {b} (q_2);
\end{tikzpicture}
    \end{center}
\end{marginfigure}
\begin{example}[Пример матрицы смежности взвешенного графа]
    \label{example:apspGraph}
    \marginnote{TODO: Как по мне подписи для примеров излишни, потому что из их содержания легко понять о чём пример. }
    Пусть дан взвешенный граф (см. рисунок \ref{gr:graph4}).
    Будем считать, что веса берутся из $\BbbR$, а решаемая задача~--- поиск кратчайших путей между вершинами.
    В таком случае естественно предположить, что для любой вершины $v_i$ существует петля $(v_i, 0, v_i)$, хоть она явно и не изображена.
    Далее, $\BbbG = (\BbbR \cup \{+\infty\}, \min)$, где $+\infty$~--- нейтральный элемент относительно операции $\min$.

    В результате мы получим следующую матрицу смежности:
    \[
        \begin{pmatrix}
            0      & -1.4   & \infty & \infty \\
            \infty & 0      & 2.2    & \infty \\
            0.5    & \infty & 0      & 1.85   \\
            \infty & \infty & -0.76  & 0
        \end{pmatrix}.
    \]
\end{example}

Таким образом, уже можно заметить, что введение моноида как абстракции позволяет достаточно унифицированным образом смотреть на различные графы и их матрицы смежности.
Далее мы увидим, что данный путь позволит решать унифицированным образом достаточно широкий круг задач, связанных с анализом путей в графах.
Но сперва мы сформулируем различные варианты задачи поиска путей в графе.
% Это необходимо для формулировки задач, на решение которых мы в конечном итоге нацелены.

\section{Обход графа в ширину}

Обход графа в ширину (Breadth-First Search, BFS)~--- это одна из фундаментальных задач анализа графов, для решения которой существует соответствующий алгоритм, изложенный в классической литературе.
%(подробнее, например, в~\cite{} или~\cite{}).
Данный алгоритм является основой для многих алгоритмов поиска в графе.

В общих чертах, задача заключается в том, чтобы начиная с некоторой заданной вершины графа (источника) обойти все достижимые из неё вершины в некотором порядке.
Шаг~--- просмотр все смежных.
И так для всего фронта.
Главное не посещать одну и ту же вершину несколько раз.

Псевдокод классического алгоритма

Пример.

Алгоритм обхода в ширину может быть переформулирован в терминах матрично-векторных операций следующим образом%
\sidenote{
    Стоит отметить, что ситуация с не менее известным обходом в глубину (Depth-First Search, DFS) более сложная: на момент написания текста не известно естественного выражения данного обхода в терминах линейной алгебры.
    Доказательство невозможности такого построения также не предъявлены.
    При этом, решения для частных случаев (деревья, ориентированные графы без циклов) предложены, например, в работе~\cite{10.1145/3315454.3329962}.}.
Пусть фронт~--- вектор размера $n$, а сам граф представлен матрицей смежности.
Тогда один шаг~--- получение нового фронта~--- это умножение текущего фронта на матрицу смежности.
Для того, чтобы отслеживать посещённые вершины нужна маска.

Псевдокод алгоритма на ЛА

\begin{example}

    Рассмотрим обход в ширину графа из примера~\ref{example:diGraph} начиная с вершины 2.
    Для обозначения текущего фронта будем использовать зелёный цвет, а для достижимых из него за один шаг~--- жёлтый, обведём вершину красным, если она посешается повторно.
    \marginnote{TODO: Как по мне, так надо просто написать каждый шаг словами, и тогда станет сильно проще сверстать}
    \begin{tabular}[t]{c | c}
        \begin{minipage}[c]{0.35\textwidth}
            \begin{tikzpicture}[]
  % Apex angle 60 degrees
  \node[isosceles triangle,
      isosceles triangle apex angle=60,
      draw=none,fill=none,
      minimum size=2cm] (T60) at (3,0){};


  \node[state,fill=yellow] (q_0) at (T60.right corner) {$0$};
  \node[state] (q_1) at (T60.left corner)  {$1$};
  \node[state,fill=green] (q_2) at (T60.apex)         {$2$};
  \node[state,fill=yellow] (q_3) [right=1.5cm of q_2]  {$3$};
  \path[->]
    (q_0) edge (q_1)
    (q_1) edge (q_2)
    (q_2) edge (q_0)
    (q_2) edge [bend left] (q_3)
    (q_3) edge [bend left] (q_2);
\end{tikzpicture}
        \end{minipage}
         &
        \begin{minipage}{0.7\textwidth}
            $\begin{aligned}
                    \emph{current\_front} & =
                    \begin{pmatrix}
                        0 & 0 & 1 & 0
                    \end{pmatrix}
                    \\
                    \emph{visited}        & =
                    \begin{pmatrix}
                        0 & 0 & 0 & 0
                    \end{pmatrix}
                    \\
                    \emph{new\_front}     & =
                    \emph{current\_front} \cdot M
                    \\
                                          & =
                    \begin{pmatrix}
                        0 & 0 & 1 & 0
                    \end{pmatrix}
                    \begin{pmatrix}
                        0 & 1 & 0 & 0 \\
                        0 & 0 & 1 & 0 \\
                        1 & 0 & 0 & 1 \\
                        0 & 0 & 1 & 0
                    \end{pmatrix}                                             \\
                                          & =
                    \begin{pmatrix}
                        1 & 0 & 0 & 1
                    \end{pmatrix}
                    \\
                    \emph{visited}        & = \emph{visited} \emph{new\_front} \\
                                          & =
                    \begin{pmatrix}
                        0 & 0 & 0 & 0
                    \end{pmatrix}
                    \begin{pmatrix}
                        0 & 0 & 1 & 0
                    \end{pmatrix}
                    \\
                    \emph{current\_front} & = \emph{visited} \emph{new\_front}
                    \\
                                          & =
                    \begin{pmatrix}
                        1 & 0 & 0 & 1
                    \end{pmatrix}
                    \begin{pmatrix}
                        0 & 0 & 1 & 0
                    \end{pmatrix}                                             \\
                                          & =
                    \begin{pmatrix}
                        1 & 0 & 0 & 1
                    \end{pmatrix}                                             \\
                \end{aligned}$
        \end{minipage}
        \\ \hline
        \begin{minipage}[c]{0.35\textwidth}
            \begin{tikzpicture}[]
  % Apex angle 60 degrees
  \node[isosceles triangle,
      isosceles triangle apex angle=60,
      draw=none,fill=none,
      minimum size=2cm] (T60) at (3,0){};


  \node[state,fill=green] (q_0) at (T60.right corner) {$0$};
  \node[state,fill=yellow] (q_1) at (T60.left corner)  {$1$};
  \node[state,fill=yellow, draw=red, line width=0.45mm] (q_2) at (T60.apex)         {$2$};
  \node[state,fill=green] (q_3) [right=1.5cm of q_2]  {$3$};
  \path[->]
    (q_0) edge (q_1)
    (q_1) edge (q_2)
    (q_2) edge (q_0)
    (q_2) edge [bend left] (q_3)
    (q_3) edge [bend left] (q_2);
\end{tikzpicture}
        \end{minipage}
         &
        $\begin{aligned}
                  & \begin{pmatrix}
                       1 & 0 & 0 & 1
                   \end{pmatrix}
                 \begin{pmatrix}
                    0 & 1 & 0 & 0 \\
                    0 & 0 & 1 & 0 \\
                    1 & 0 & 0 & 1 \\
                    0 & 0 & 1 & 0
                \end{pmatrix}    \\ &=
                 \begin{pmatrix}
                    0 & 1 & 1 & 0
                \end{pmatrix}
             \end{aligned}$
        \\ \hline
        \begin{minipage}[c]{0.35\textwidth}
            \begin{tikzpicture}[]
  % Apex angle 60 degrees
  \node[isosceles triangle,
      isosceles triangle apex angle=60,
      draw=none,fill=none,
      minimum size=2cm] (T60) at (3,0){};


  \node[state] (q_0) at (T60.right corner) {$0$};
  \node[state,fill=green] (q_1) at (T60.left corner)  {$1$};
  \node[state,fill=yellow, draw=red, line width=0.45mm] (q_2) at (T60.apex)         {$2$};
  \node[state] (q_3) [right=1.5cm of q_2]  {$3$};
  \path[->]
    (q_0) edge (q_1)
    (q_1) edge (q_2)
    (q_2) edge (q_0)
    (q_2) edge [bend left] (q_3)
    (q_3) edge [bend left] (q_2);
\end{tikzpicture}
        \end{minipage}
         &
        $
            \begin{pmatrix}
                0 & 1 & 0 & 0
            \end{pmatrix}
            \begin{pmatrix}
                0 & 1 & 0 & 0 \\
                0 & 0 & 1 & 0 \\
                1 & 0 & 0 & 1 \\
                0 & 0 & 1 & 0
            \end{pmatrix} =
            \begin{pmatrix}
                0 & 0 & 1 & 0
            \end{pmatrix}
        $
    \end{tabular}

\end{example}


multiple-source BFS. Тот же обход в ширину, только источников несколько и надо помнить, какая из вершин из какого источника достижима. Постановка задачи. Решение через линейную алгебру~\sidecite{9286186} Уже не вектор, а матрица: храним информацию про каждую стартовую вершину отдельно.

Псевдокод алгоритма на ЛА

Пример.

\begin{minipage}[c]{0.35\textwidth}
    \begin{tikzpicture}[]
  % Apex angle 60 degrees
  \node[isosceles triangle,
      isosceles triangle apex angle=60,
      draw=none,fill=none,
      minimum size=2cm] (T60) at (3,0){};


  \node[state] (q_0) at (T60.right corner) {$0$};
  \node[state,fill=green] (q_1) at (T60.left corner)  {$1$};
  \node[state,fill=yellow] (q_2) at (T60.apex)         {$2$};
  \node[state,fill=green] (q_3) [right=1.5cm of q_2]  {$3$};
  \path[->]
    (q_0) edge (q_1)
    (q_1) edge (q_2)
    (q_2) edge (q_0)
    (q_2) edge [bend left] (q_3)
    (q_3) edge [bend left] (q_2);
\end{tikzpicture}
\end{minipage}
\begin{minipage}{0.7\textwidth}
\end{minipage}


\begin{minipage}[c]{0.35\textwidth}
    \begin{tikzpicture}[]
  % Apex angle 60 degrees
  \node[isosceles triangle,
      isosceles triangle apex angle=60,
      draw=none,fill=none,
      minimum size=2cm] (T60) at (3,0){};


  \node[state,fill=yellow] (q_0) at (T60.right corner) {$0$};
  \node[state] (q_1) at (T60.left corner)  {$1$};
  \node[state,fill=green] (q_2) at (T60.apex)         {$2$};
  \node[state,fill=yellow, draw=red, line width=0.45mm] (q_3) [right=1.5cm of q_2]  {$3$};
  \path[->]
    (q_0) edge (q_1)
    (q_1) edge (q_2)
    (q_2) edge (q_0)
    (q_2) edge [bend left] (q_3)
    (q_3) edge [bend left] (q_2);
\end{tikzpicture}
\end{minipage}
\begin{minipage}{0.7\textwidth}
\end{minipage}


\section{Задачи поиска путей}

Одна из классических задач анализа графов~--- это задача поиска путей между вершинами с различными ограничениями.

При этом возможны различные постановки задачи.
С одной стороны, постановки различаются тем, что именно мы хотим получить в качестве результата. Здесь наиболее частыми являются следующие варианты.

\begin{itemize}
    \item Наличие хотя бы одного пути, удовлетворяющего ограничениям, в графе.
          В данном случае не важно, между какими вершинами существует путь, важно лишь наличие его в графе.
    \item Наличие пути, удовлетворяющего ограничениям, между некоторыми вершинами: задача достижимости.
          При данной постановке задачи, нас интересует ответ на вопрос достижимости вершины $v_i$ из вершины $v_j$ по пути, удовлетворяющему ограничениям.
          Такая постановка требует лишь проверить существование пути, но не его предоставления в явном виде.
    \item Поиск одного пути, удовлетворяющего ограничениям: необходимо не только установить факт наличия пути, но и  предъявить его.
          При этом часто подразумевается, что возвращается любой путь, являющийся решением, без каких-либо дополнительных ограничений.
          Хотя, например, в некоторых задачах дополнительное требование простоты или наименьшей длины выглядит достаточно естественным.
    \item Поиск всех путей: необходимо предоставить все пути, удовлетворяющие заданным ограничениям.
\end{itemize}

С другой стороны, задачи могут различаться ещё и тем, как фиксируются множества стартовых и конечных вершин.
Здесь возможны следующие варианты:
\begin{itemize}
    \item от одной вершины до всех,
    \item между всеми парами вершин,
    \item между фиксированной парой вершин,
    \item между двумя множествами вершин $V_1$ и $V_2$, что подразумевает решение задачи для всех $(v_i,v_j) \in V_1 \times V_2$.
\end{itemize}

Стоит отметить, что последний вариант является самым общим и остальные~--- лишь его частные случаи.
Однако этот вариант часто выделяют отдельно, подразумевая, что остальные, выделенные, варианты в него не включаются.

В итоге, перебирая возможные варианты желаемого результата и способы фиксации стартовых и финальных вершин, мы можем сформулировать достаточно большое количество задач.
Например, задачу поиска всех путей между двумя заданными вершинами, задачу поиска одного пути от фиксированной стартовой вершины до каждой вершины в графе, или задачу достижимости между всеми парами вершин.

Часто поиск путей сопровождается изучением их свойств, что далее приводит к формулированию дополнительных ограничений на пути в терминах этих свойств.
Например, можно потребовать, чтобы пути были простыми или не проходили через определённые вершины.
Один из естественных способов описывать свойства и, как следствие, задавать ограничения~--- это использовать ту алгебраическую структуру, из которой берутся веса рёбер графа%
\sidenote{На самом деле здесь наблюдается некоторая двойственность.
    С одной стороны, действительно, удобно считать, что свойства описываются в терминах некоторой заданной алгебраической структуры.
    Но, вместе с этим, структура подбирается исходя из решаемой задачи.}.

Предположим, что дан граф $\mbfscrG = \langle V, E, L\rangle $, где $L = (S, \oplus, \otimes)$~--- это полукольцо.
Тогда изучение свойств путей можно описать следующим образом:
\begin{equation}
    \label{eq:algPathProblem}
    \left\{(v_i, v_j, c) \mid \exists v_i \pi v_j, c = \bigoplus_{\forall v_i \pi v_j} \bigotimes_{(u, l, v) \in \pi } l\right\}.
\end{equation}
\marginnote{TODO: В общем случае меня смущает forall множестве, по которому происходит бинарная операция}

Иными словами, для каждой пары вершин, для которой существует хотя бы один путь, их соединяющий, мы агрегируем (с помощью операции $\oplus$ из полукольца) информацию обо всех путях между этими вершинами.
При этом информация о пути получается как свёртка меток рёбер пути с использованием операции $\otimes$\sidenote{Заметим, что детали свёртки вдоль пути зависят от свойств полукольца (и от решаемой задачи).
    Так, если полукольцо коммутативно, то нам не обязательно соблюдать порядок рёбер.
    В дальнейшем мы увидим, что данные особенности полукольца существенно влияют на особенности алгоритмов решения соответствующих задач.}.

Естественным требованием (хотя бы для прикладных задач, решаемых таким способом) является существование и конечность указанной суммы.
На данном этапе мы не будем касаться того, какие именно свойства полукольца могут нам обеспечить данное свойство, однако в дальнейшем будем считать, что оно выполняется.
Более того, будем стараться приводить частные для конкретной задачи рассуждения, показывающие, почему это свойство выполняется в рассматриваемых в задаче ограничениях.

Описанная выше задача общего вида называется анализом свойств путей алгебраическими методами (Algebraic Path Problem)~\sidecite{Baras2010PathPI} и предоставляет общий способ для решения широкого класса прикладных задач%
\sidenote{В работе \enquote{Path Problems in Networks}~\cite{Baras2010PathPI} собран действительно большой список прикладных задач с описанием соответствующих полуколец.
    Сводная таблица на страницах 58--59 содержит 29 различных прикладных задач и соответствующих полуколец.}.
Наиболее известными являются такие задачи, как построение транзитивного замыкания графа и поиск кратчайших путей (All Pairs Shortest Path или APSP).
Далее мы подробнее обсудим эти две задачи и предложим алгоритмы их решения.

\section{Алгоритм Флойда-Уоршелла}

Наиболее естественным образом решение обсуждаемых выражается в терминах операций над матрицей смежности исходного графа.
Поэтому предположим, что исходный граф задан матрицей над моноидом $\BbbG = (S,\oplus)$.

Как мы видели ранее, операция $\oplus$ позволяет нам агрегировать информацию по всем параллельным рёбрам.
Ровно она же и будет агрегировать информацию по всем путям между двумя вершинами%
\sidenote{Вообще говоря, работая с матрицей смежности мы не видим разницу между путём и ребром, так как любая запись в матрице смежности в ячейке $[i, j]$ говорит нам только о том, что вершины $i$ и $j$ связаны и эта связь обладает некоторым свойством (значение в ячейке), и ничего не говорит о том, как эта связь устроена.}.
Таким образом, осталось сконструировать операцию, отвечающую за агрегацию информации вдоль пути.
Здесь мы будем исходить из того, что новый путь может быть получен из двух подпутей, а свойство нового пути зависит только от свойств исходных подпутей.

Таким образом, дополнительная операция, обозначим её $\otimes: S \times S \to S$%
\sidenote{При первом рассмотрении такой выбор кажется контринтуитивным.
    Действительно, ведь при соединении путей мы как бы \enquote{складываем} их веса.
    Но при более детальном анализе поведения этой операции, в частности, относительно нейтрального элемента, становится понятно, что она ведёт себя очень похоже на умножение.
    Вероятно, стоит обратить внимание на операцию конкатенации, которая, с одной стороны, \enquote{делает то, что нам нужно}, а с другой, (и неспроста) часто обозначается $\cdot$.}%
, должна вести себя следующим образом. Пусть $S$~--- носитель моноида, $\Bbbzero \in S$~--- нейтральный элемент относительно $\oplus$.
\begin{itemize}
    \label{itm:otimesIntro}
    \item $s_1 \otimes s_2 = s_3$, $s_i \in S$, $s_i \neq \Bbbzero$: если существует путь $i \pi j$ со свойством $s_1$ и путь $j \pi k$ со свойством $s_2$, то существует путь $i \pi k$ со свойством $s_3$.
    \item $s \otimes \Bbbzero = \Bbbzero$: если существует путь $i \pi j$ со свойством $s$ и не существует пути $j \pi k$, то не существует и пути $i \pi k$.
    \item $\Bbbzero \otimes s = \Bbbzero$: если не существует пути $i \pi j$ и существует путь $j \pi k$ со свойством $s$, то не существует и пути $i \pi k$.
    \item $\Bbbzero \otimes \Bbbzero = \Bbbzero$: если не существует пути $i \pi j$ и не существует пути $j \pi k$, то не существует и пути $i \pi k$.
\end{itemize}

Новую операцию добавим к моноиду и получим новую алгебраическую структуру $\BbbG' = (S, \oplus,\otimes)$.
Данная структура является коммутативным моноидом по сложению (по построению) с нейтральным элементом $\Bbbzero$.
Из построения $\otimes$ видно, что $\Bbbzero$ является аннигилятором.
Ничего более про операцию $\otimes$, а значит и про $\BbbG'$ мы сказать, исходя из построения, не можем.
Классический фреймворк для решения algebraic path problem подразумевает, что $\BbbG'$ является полукольцом, однако далее мы увидим, что существуют задачи, в которых $\otimes$, например, не является ассоциативной%
\sidenote{Такой будет рассматриваемая в данной работе задача достижимости с ограничениями в терминах формальных языков.
    Другие примеры можно найти в уже упоминавшейся работе~\cite{Baras2010PathPI}.}.
А значит, согласно нашему определению, $\BbbG'$ полукольцом не является.

Теперь, когда построена алгебраическая структура, обеспечивающая вычисление формулы~
\ref{eq:algPathProblem}, мы можем предложить алгоритм вычисления этой формулы и данным алгоритмом в интересующих нас частных случаях будет являться алгоритм Флойда-Уоршелла~\sidecite{Floyd1962, Bernard1959, Warshall1962}.
Псевдокод алгоритма представлен на листинге~\ref{lst:algoFloydWarshall}, а его сложность $O(n^3)$.
Он практически дословно основан на описанной выше идее сборки путей из двух подпутей: тройной вложенный цикл перебирает все возможные разбиения пути на две части, а в строке 7 как раз и происходит вычисление формулы~\ref{eq:algPathProblem}.

Необходимо обратить внимание на несколько вещей.
Первая~--- порядок обхода.
Внешний цикл перебирает возможные точки разбиения (хотя мог бы, например, перебирать начальные вершины) для того, чтобы гарантировать правильный порядок вычисления подпутей (информация ни о каких подпутях не будет получена после того, как они поучаствовали в построении более длинного пути).
Вторая~--- количество итераций.
В данном случае мы ограничились тройным вложенным циклом от 0 до $n$ и для наших задач этого будет достаточно, однако, как доказательство этого факта, так и построение аналогичного алгоритма для других задач требует аккуратного анализа решаемой задачи и последующего доказательства корректности построенного алгоритма.

% \begin{algorithm}
%   \floatname{algorithm}{Listing}
%   \begin{algorithmic}[1]
%     \caption{Алгоритм Флойда-Уоршелла}
%     \label{lst:algoFloydWarshall}
%     \Function{FloydWarshall}{$\mbfscrG$}
%     \State{$M \gets$ матрица смежности $\mbfscrG$}
%     \Comment{Матрица над $\BbbG=(S,\oplus,\otimes)$}
%     \State{$n \gets |V(\mbfscrG)|$}
%     \For{k = 0; k < n; k++}
%     \For{i = 0; i < n; i++}
%     \For{j = 0; j < n; j++}
%     \State{$M[i,j] \gets M[i,j] \oplus (M[i,k] \otimes M[k,j])$}
%     \EndFor
%     \EndFor
%     \EndFor
%     \State \Return $M$
%     \EndFunction
%   \end{algorithmic}
% \end{algorithm}

Хотя изначально данный алгоритм был предложен для решения задачи о кратчайших путях, при абстрагировании алгебраической структуры он превращается в алгоритм решения целого ряда задач.
В частности~--- нахождения транзитивного замыкания.
Так, если возьмём тропическое полукольцо $(\BbbR_{+\infty}, \min, +)$, то получим алгоритм для поиска кратчайших путей.
Если же возьмём булево полукольцо, то получим алгоритм для построения транзитивного замыкания графа.

\begin{marginfigure}
    \caption{Не полный граф}
    \label{gr:graph2nottc}
    \begin{center}
        \begin{tikzpicture}[]
  % Apex angle 60 degrees
  \node[isosceles triangle,
      isosceles triangle apex angle=60,
      draw=none,fill=none,
      minimum size=2cm] (T60) at (3,0){};


  \node[state] (q_0) at (T60.right corner) {$0$};
  \node[state] (q_1) at (T60.left corner)  {$1$};
  \node[state] (q_2) at (T60.apex)         {$2$};
  \node[state] (q_3) [right=1.5cm of q_2]  {$3$};
  \path[->]
    (q_0) edge[left]  node {a} (q_1)
          edge[bend right, below] node {S} (q_2)
    (q_1) edge[above] node {a} (q_2)
          edge[bend left, above, out=30, in=135] node {S} (q_3)
    (q_2) edge[below] node {a} (q_0)
          edge[bend left, above]  node {b, \textbf{S}} (q_3)
    (q_3) edge[bend left, below]  node {b} (q_2);
\end{tikzpicture}
    \end{center}
\end{marginfigure}
\begin{marginfigure}
    \caption{Полный граф}
    \label{gr:graph5}
    \begin{center}
        \begin{tikzpicture}[]
  % Apex angle 60 degrees
  \node[isosceles triangle,
      isosceles triangle apex angle=60,
      draw=none,fill=none,
      minimum size=2cm] (T60) at (3,0){};


  \node[state] (q_0) at (T60.right corner) {$0$};
  \node[state] (q_1) at (T60.left corner)  {$1$};
  \node[state] (q_2) at (T60.apex)         {$2$};
  \node[state] (q_3) [right=1.5cm of q_2]  {$3$};
  \path[->]
    (q_0) edge[loop below] node {} ()
          edge  node {} (q_1)
          edge[bend right] node {} (q_2)
          edge[bend right = 55]  node {} (q_3)
    (q_1) edge[loop above] node {} ()
          edge[bend right] node {} (q_0)
          edge  node {} (q_2)
          edge[bend left = 55, above]  node {} (q_3)
    (q_2) edge[loop below] node {} ()
          edge[bend right] node {} (q_1)
          edge  node {} (q_0)
          edge[bend left, above]  node {} (q_3)
    (q_3) edge[loop above] node {} ()
          edge[bend left, below]  node {} (q_2)
          edge[bend left, below]  node {} (q_0)
          edge[bend right, below]  node {} (q_1)
    ;
\end{tikzpicture}
    \end{center}
\end{marginfigure}
\begin{example}[Транзитивное замыкание графа]
    \label{exmpl:transitiveClosure}
    Пусть дан граф (см. рисунок~\ref{gr:graph2nottc}).
    Его матрица смежности:
    \[
        M =
        \begin{pmatrix}
            0 & 1 & 0 & 0 \\
            0 & 0 & 1 & 0 \\
            1 & 0 & 0 & 1 \\
            0 & 0 & 1 & 0
        \end{pmatrix}.
    \]

    Здесь мы считаем, что отношение достижимости не рефлексивно: все диагональные элементы матрицы $M$ равны 0.

    Воспользовавшись алгоритмом из листинга~\ref{lst:algoFloydWarshall}, специализированного на случай булева полукольца, можно получить следующую матрицу смежности.
    \[
        M' =
        \begin{pmatrix}
            1 & 1 & 1 & 1 \\
            1 & 1 & 1 & 1 \\
            1 & 1 & 1 & 1 \\
            1 & 1 & 1 & 1
        \end{pmatrix}.
    \]

    А значит, транзитивным замыканием исходного графа является полный граф, изображенный на рисунке~\ref{gr:graph5}.
\end{example}

Заметим, что рефлексивность отношения (значения элементов на главной диагонали матрицы смежности) непосредственно связана с особенностями решаемой задачи.
Например, если говорить о кратчайших расстояниях, то кажется естественным считать расстояние от вершины до самой себя равной нулю.
Однако для задачи транзитивного замыкания это уже не столь естественное предположение и часто отдельно говорят о рефлексивно-транзитивном замыкании, которое отдельно явным образом привносит рефлексивность (петли, диагональные элементы матрицы смежности).

Аналогичным образом, используя данный алгоритм, но уже для тропического полукольца, можно решить задачу о поиске кратчайших путей для графа из примера~\ref{example:apspGraph}.

Идеи, заложенные в алгоритме Флойда-Уоршелла, а также возможность абстрагировать его, помогут нам в дальнейшем предложить алгоритм для задачи достижимости с ограничениями в терминах формальных языков.

%% FIXME: Исправить раздел
% \section{Анализ путей в графе и линейная алгебра}

% В данной главе мы рассмотрим некоторые связи%
% \sidenote{Связь между графами и линейной алгеброй~--- обширная область, в которой можно даже выделить отдельные направления, такие как спектральная теория графов.
%     С точки зрения практики данная связь также подмечена давно и более полно с ней можно ознакомиться, например, в работах~\cite{doi:10.1137/1.9780898719918, Davis2018Algorithm9S}.
%     Кроме этого, много полезной информации можно найти на сайте \url{https://graphblas.github.io/GraphBLAS-Pointers/}.}
% между графами и операциями над ними и матрицами и операциями над матрицами.

% Как мы видели, достаточно естественное представление графа~--- это его матрица смежности.
% Далее можно заметить некоторое сходство между определением матричного умножения~\ref{def:MxM} и мыслями, которыми мы руководствовались, вводя операцию $\otimes$ (\ref{itm:otimesIntro}).
% Действительно, пусть есть матрица $M$ размера $n \times n$ над $\BbbG = (S, \oplus, \otimes)$%
% \sidenote{Здесь мы уже сталкиваемся с тем, что могут иметь смысл относительно экзотические алгебраические структуры.
%     Действительно, как мы выяснили, матрица смежности может быть определена на чем-то более бедным, чем полукольцо, а матричное умножение мы определяли над полукольцом.
%     Но если задуматься, то и для определения произведения матриц полукольцо вовсе необязательно, достаточно более бедной структуры.}~--- матрица смежности графа.
% Умножим её саму на себя: вычислим $M'= M \cdot M$.
% Тогда, по~\ref{def:MxM}, $M'[i, j] = \bigoplus_{l \in [0 \rng n-1]} M[i, l] \otimes M[l, j]$.
% Размер $M'$ также $n \times n$.
% То есть $M'$ задаёт такой граф, что в нём будет путь со свойством, являющимся агрегацией свойств всех путей, составленных из двух подпутей в исходном графе.
% А именно, если в исходном графе есть путь из $i$ в $l$ с некоторым свойством (его значение хранится в $M[i, l]$), и был путь из $l$ в $j$ (его значение хранится в $M[l,j]$), то в новом графе свойство $M[i, l] \otimes M[l, j]$ аддитивно (используя $\oplus$) учтётся в свойстве пути из $i$ в $j$.

% Таким образом, произведение матриц смежности соответствует обработке информации о путях интересующим нас образом.
% Это наблюдение позволяет предложить решение задач анализа свойств путей, основанное на операциях над матрицами.
% Рассмотрим такое решение для задачи о кратчайших путях.

% Пусть $D_k$~--- матрица кратчайших путей, состоящих не более чем из $k$ рёбер.
% То есть $D_k[i, j]$~--- это длина кратчайшего пути из вершины $i$ в вершину $j$, такого, что он состоит не более чем из $k$ ребер.
% Если такого пути нет, то $D_k[i, j] = +\infty$.

% Таким образом, $D_1 = M$, где $M$~--- это матрица смежности исходного графа, а решением APSP является $D_{n-1}$, вычисляемая с помощью следующего рекуррентного соотношения:
% \begin{gather*}
%     D(1) = M \\
%     D(2) = D(1) \cdot M = M^2 \\
%     D(3) = D(2) \cdot M = M^3 \\
%     \vdots \\
%     D(n - 1) = D(n - 2) \cdot M = M^{(n - 1)}
% \end{gather*}

% Здесь мы пользуемся той особенностью задачи, что кратчайший путь в ориентированном графе (без отрицательных циклов) не может быть длиннее $n$%
% \sidenote{Раз отрицательных циклов нету, то проходить через одну вершину, коих $n$, больше одного раза бессмысленно.}.
% Более того, мы пользуемся тем, что используемая структура именно полукольцо, а исходное отношение рефлексивно%
% \sidenote{
%     Тот факт, что в любой вершине есть петля с весом 0, а 0~--- нейтральный для $\otimes$, позволяет не суммировать матрицы.
%     Итоговая матрица содержит данные о путях длины \emph{не больше чем} вычисляемая степень, хотя должна бы содержать данные о путях ровно и только такой длины.
%     Предлагается самостоятельно исследовать данный феномен.}.

% Таким образом, решение APSP сведено к произведению матриц над тропическим полукольцом, однако вычислительная сложность решения слишком большая: $O(n K(n))$, где $K(n)$~--- сложность алгоритма умножения матриц.

% Чтобы улучшить сложность, заметим, что, например, $D_3$ вычислять не обязательно, так как можно сразу вычислить $D_4$ как $D_2 \cdot D_2$.

% В итоге получим следующую последовательность вычислений:
% \begin{gather*}
%     D_1 = M \\
%     D_2 = M^2 = M \cdot M \\
%     D_4 = M^4 = M^2 \cdot M^2 \\
%     D_8 = M^8 = M^4 \cdot M^4 \\
%     \vdots \\
%     D_{2^{\log(n-1)}} = M^{2^{\log(n-1)}} = M^{2^{\log(n-1)} - 1} \cdot M^{2^{\log(n-1)} - 1} \\
%     D_{n-1} = D_{2^{\log(n-1)}}
% \end{gather*}

% Теперь вместо $n$ итераций нам нужно $\log{n}$, а итоговая сложность решения~--- $O(\log{n} K(n))$%
% \sidenote{
%     Заметим, что это оценка для худшего случая.
%     На практике при использовании данного подхода можно прекращать вычисления как только при двух последовательных шагах получились одинаковые матрицы ($D_i = D_{i-1}$).
%     Это приводит нас к понятию \emph{неподвижной точки}, обсуждение которого лежит за рамками повествования.}.
% Данный алгоритм называется \emph{repeated squaring}%
% \sidenote{Пример решения APSP с помощью repeated squaring: \url{http://users.cecs.anu.edu.au/~Alistair.Rendell/Teaching/apac_comp3600/module4/all_pairs_shortest_paths.xhtml}}.
% Здесь мы предполагаем, что $n-1 = 2^k$ для какого-то $k$.
% На практике такое верно далеко не всегда.
% Если это условие не выполняется, то необходимо взять ближайшую сверху степень двойки%
% \sidenote{
%     Кажется, что это приведёт к избыточным вычислениям.
%     Попробуйте оценить, на сколько много лишних вычислений будет сделано в худшем случае.}.

% Таким образом, APSP сводится к умножению матриц над полукольцом, что, к сожалению, не позволяет этим путём получить истинно субкубический алгоритм для задачи.
% Тем не менее, позволяет получить слегка субкубический.
% Приведем некоторые работы по APSP для ориентированных графов с вещественными весами (здесь $n$~--- количество вершин в графе), по которым можно более детально ознакомиться как с историей вопроса, так и с текущими результатами:
% \begin{itemize}
%     \item M.L. Fredman (1976)~--- $O(n^3(\log \log n / \log n)^\frac{1}{3})$~--- использование дерева решений~\sidecite{FredmanAPSP1976}
%     \item W. Dobosiewicz (1990)~--- $O(n^3 / \sqrt{\log n})$~--- использование операций на Word Random Access Machine~\sidecite{Dobosiewicz1990}
%     \item T. Takaoka (1992)~--- $O(n^3 \sqrt{\log \log n / \log n})$~--- использование таблицы поиска~\sidecite{Takaoka1992}
%     \item Y. Han (2004)~--- $O(n^3 (\log \log n / \log n)^\frac{5}{7})$~\sidecite{Han2004}
%     \item T. Takoaka (2004)~--- $O(n^3 (\log \log n)^2 / \log n)$~\sidecite{Takaoka2004}
%     \item T. Takoaka (2005)~--- $O(n^3 \log \log n / \log n)$~\sidecite{Takaoka2005}
%     \item U. Zwick (2004)~--- $O(n^3 \sqrt{\log \log n} / \log n)$~\sidecite{Zwick2004}
%     \item T.M. Chan (2006)~--- $O(n^3 / \log n)$~--- многомерный принцип \enquote{разделяй и властвуй}~\sidecite{Chan2008}
% \end{itemize}

% Вопрос же о истинно субкубических алгоритмах решения APSP всё ещё открыт~\sidecite{Chan2010} и активно обсуждается в научном сообществе.

% Аналогичным образом можно свести транзитивное замыкание к матричным операциям.
% Предлагаем проделать это самостоятельно, заодно обратив внимание на важность рефлексивности (в примере~\ref{exmpl:transitiveClosure} её нет).

% Заметим, что оптимизация, связанная с возведением в квадрат возможна только при ассоциативности произведения матриц, что зависит от свойств алгебраической структуры, над которой построены матрицы.
% Для полукольца такая оптимизация законна, однако в некоторых случаях её применить нельзя.
% Таким случаем как раз и будет рассматриваемая в нашей работе задача.
% Поэтому построение алгоритма её решения через операции над матрицами, хотя и будет основано на указанных выше соображениях, но будет сопряжено с некоторыми трудностями.

% %\section{Вопросы и задачи}
% %\begin{enumerate}
% %  \item Реализуйте абстракцию полукольца, позволяющую конструировать полукольца с произвольными операциями.
% %  \item Реализуйте алгоритм произведения матриц над произвольным полукольцом. Используйте результат решения предыдущей задачи.
% %  \item Используя результаты предыдущих задач, реализуйте алгоритм построения транзитивного замыкания через произведение матриц.
% %  \item Используя результаты предыдущих задач, реализуйте алгоритм решения задачи APSP для ориентированного через произведение матриц.
% %  \item Используя существующую библиотеку линейной алгебры для CPU, решите задачу построения транзитивного замыкания графа.
% %  \item Используя существующую библиотеку линейной алгебры для CPU, решите задачу APSP для ориентированного графа.
% %  \item Используя существующую библиотеку линейной алгебры для GPGPU, решите задачу построения транзитивного замыкания графа.
% %  \item Используя существующую библиотеку линейной алгебры для GPGPU, решите задачу APSP для ориентированного графа.
% %  \item Сравните произволительность решений предыдущих задач
% %\end{enumerate}
